% This file contains the content for the Introduction
\unnumberedformat	    % Change formatting to that of "Introduction" section
\chapter{Introduction} 	% Do not modify section title
%% Modify below this line %%

The Academy Color Encoding System is a free, open, device-independent color management and image interchange system that can be applied to almost any current or future workflow. It was developed by hundreds of the industry's top scientists, engineers, and end users, working together under the auspices of the Academy of Motion Picture Arts and Sciences.

The primary color encoding in the Academy Color Encoding System (ACES) is the Academy Color Encoding Specification (ACES2065-1).  Academy Color Encoding Specification is standardized in SMPTE ST 2065-1:2012 \cite{SMPTE20651}.  As part of the specification, the encoding primaries and white point were specified as CIE xy chromaticity coordinates to allow for the transformation of ACES2065-1 RGB values to and from other color spaces including CIE XYZ.  Though the CIE xy chromaticity coordinates of encoding red, green, blue and white primaries are only one factor important to unambiguous color interchange\cite{giorgianni}, their specification is required for the calculation of a normalized primary matrix used in color space transformations \cite{smpteRP1997}. The white point used in ACES2065-1 was later adopted for use in other ACES encodings such as ACEScg, ACEScc, ACEScct, etc \cite{ACEScg,ACEScc,ACEScct}. For brevity and inclusiveness, the white point used in the various encodings will be referred to as "the ACES white point" throughout the remainder of this document unless more specificity is required.

The derivation of the ACES white point chromaticity coordinates outlined in this document is intended to help technical users of the ACES system calculate transformations to and from the various ACES encodings in as accurate a manner as possible.  The white point of the ACES encodings does not limit the choice of sources that may be used to photograph or generate source images, nor does it dictate the white point of the reproduction. Using various techniques beyond the scope of this document, the chromaticity of the reproduction of equal ACES2065-1 red, green and blue values (ACES2065-1 \rgbequal) may match the chromaticity of the ACES white point, the display calibration white point, or any other white point preferred for technical or aesthetic reasons

ACES technical documentation is available via ACEScentral.com and oscars.org/aces for product developers wishing to implement ACES concepts and specifications into their products and for workflow/pipeline designers to use ACES concepts and ACES-enabled products for their productions.