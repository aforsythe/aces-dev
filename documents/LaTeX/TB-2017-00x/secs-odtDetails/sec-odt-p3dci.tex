% Section Start
\section[P3-DCI]{\shortName{}}
\label{sec:odt-details-\id}

%% Summary
\subsection{Summary}
\label{subsec:summary-\id}

It is common in the digital intermediate process (DI) to color correct motion pictures and episodic television shows while displaying the images using a DCI compliant digital cinema projector. DCI compliant digital cinema projectors have a simplified setup using a projector configuration file (PCF) that contains all the relevant projector settings and can often be loaded at the press of a button. The color calibration the projector is specified in the PCF using a Target Color Gamut Document (TCGD) that specifies the calibration color primaries and white point chromaticity aims.  The most common PCF used in motion picture and television production is the ``DCI-P3'' PCF. Using this PCF, the projector will be configured such that equal red, green, and blue projector code values will produce the chromaticity x=0.3140 y=0.3510 on the screen. With the projector configured in this manner it is recommended that the ACES 1.0 ODT with the transformID \transformID{} be used.

%% Transform Identifiers
\subsection{Transform Identifiers} 
\label{subsec:odt-ident-\id}
\idTable{1.5}{3}

%% Recommended Display Setup
\subsection{Recommended Display and Setup}
\label{subsec:setup-\id}

\begin{table}[ht!]
    \centering
        \begin{tabular}{|p{1.5in}|p{3in}|}
            \hline
            \textbf{Parameter} 		& 	\textbf{Setting} 				 		\\ \hline
            Display Type 			&	Digital Cinema compliant projector 		\\ \hline
            Display Dynamic Range 	& 	$\geq$ 2,000:1 to $\sim$10,000:1 		\\ \hline
            Display Max Luminance 	& 	48 cd/m$^2$								\\ \hline
            TCGD 					& 	?? DCI P3 lookup name?? 				\\ \hline %TODO Find P3 DCI TCGD Filename
            Signal 					&	RGB 4:4:4 Full range 					\\ \hline
            Viewing Environment 	& 	Dark 									\\ \hline
            Bit Depth 				& 	12-bit 									\\ \hline 
    	\end{tabular}
    \caption{Display Setup: \shortName{}} 
    \label{tab:setup-\id}
\end{table}

%% Notes
\subsection{Notes}
\label{subsec:notes-\id}

Using the ``DCI-P3'' PCF, the projector will be configured such that equal red, green, and blue display code values will produce the chromaticity x=0.3140 y=0.3510 on the screen. However, the \transformID{} transform is configured such that neutral ACES source file values (ACES R=G=B) will produce non-equal projector code values. The chromaticity of produced on screen by those non-equal projector code values will be x=0.32168 y=0.33767 (aka D60) 

It's important to note that the image on projection screen may loo distinctly less green then some workflows that utilize a projector setup with the ``DCI-P3'' PCF. This will also be reflected on the color corrector scopes when neutral ACES values sent through the \transformID{} transform. (Figure \ref{fig:acesSource-p3dci}, \ref{fig:hist-p3dci}, \ref{fig:parade-p3dci}, \ref{fig:wf-p3dci}, \ref{fig:vect-p3dci}) For instance neutral ACES values processed through \transformID{} will not have equal levels o the waveform, nor will they land in the middle of the vector scope. This behavior was intentional. The image may also have a distinctly magenta cast on a computer monitor such as the on used for the color corrector user interface if that monitor i calibrated to a D65 white point. (Figure \ref{fig:cv-p3dci}) Although not noted in the name of this ODT, the mimics the behavior found in other ODTs include in ACES 1.0 and labeled ``D60 sim''. Due to this ``D60 sim'' behavior the maximum output screen luminance of neutral ACES values will b slightly less than the maximum luminance produced by projector cod values red = 1, green = 1, blue = 1 (e.g.~48 nits) 

When using the correct projector setup and corresponding ODT, the image on the projector screen will match nearly exactly in Application \ref{sec:odt-details-p3dci} an Application \ref{sec:odt-details-p3d60} 

\odtScreenshots{Projector code values as displayed on a D65 calibrated computer monitor}

%% Test Values
\subsection{Test Values}
\testValuesSubSec{}

