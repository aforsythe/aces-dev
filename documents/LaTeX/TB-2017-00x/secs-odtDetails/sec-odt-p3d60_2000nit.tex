% Section Start
\section[P3-D60 ST2084 (2000 nits)]{\shortName{\id}}
\label{sec:odt-details-\id}

%% Summary
\subsection{Summary}
\label{subsec:summary-\id}

In HDR content mastering applications it is sometimes common to use an HDR reference monitor configured with a peak luminance of \nits{2000}, P3 device primaries, and a white point chromaticity of \whitepoint{aces}, an EOTF conforming to SMPTE ST2084 (aka PQ).  The ODT with the transformID \transformID{\id} is recommended for this application.

%% Transform Identifiers
\subsection{Transform Identifiers} 
\label{subsec:odt-ident-\id}
\idTable{1.5}{3.25}

%% Recommended Display Setup
\subsection{Recommended Display and Setup}
\label{subsec:setup-\id}

\begin{table}[ht!]
    \centering
        \begin{tabular}{|p{1.5in}|p{3in}|}
            \hline
            \textbf{Parameter} 		& 	\textbf{Setting} 				 		\\ \hline
            Display Type 			&	HDR Reference Monitor					\\ \hline
            Display Max Luminance 	& 	\nits{2000}								\\ \hline
            Primaries	 			& 	P3										\\ \hline
            White Point	 			& 	D60 (\whitepoint{aces})					\\ \hline
            EOTF					& 	SMPTE ST2084		 					\\ \hline
            Signal 					&	RGB 4:4:4 (Full range or Legal Range)	\\ \hline
            Viewing Environment 	& 	Dim Surround 							\\ \hline
            Bit Depth 				& 	10 or 12-bit	 						\\ \hline 
    \end{tabular}
    \caption{Display Setup: \protect\shortName{\id}} 
    \label{tab:setup-\id}
\end{table}

%% Notes
\subsection{Notes}
\label{subsec:notes-\id}

\lipsum[1-2] %TODO: Write P3D60 2000nit Notes

\odtScreenshots{Projector code values as displayed on a D$_{65}$ calibrated computer monitor}

%% Test Values
\subsection{Test Values}
\label{subsec:testValues-\id}

\testValuesSubSec{}
